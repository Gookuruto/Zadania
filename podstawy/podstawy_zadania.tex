\documentclass[11pt]{article}
\usepackage[T1]{fontenc}
\usepackage[polish]{babel}
\usepackage{amsmath}
\usepackage{hyperref}
%\usepackage{amssymb}
%\topmargin -2cm
%\textheight 23cm
%\oddsidemargin -1cm
%\evensidemargin -1cm
%\textwidth 17.5cm
\usepackage[lmargin=1cm,rmargin=1.5cm,tmargin=1cm,bmargin=1cm]{geometry}
\usepackage{natbib}
\usepackage{graphicx}
\usepackage{listings,lstautogobble}

\lstset{language=Python,
	basicstyle=\scriptsize\ttfamily,
	stringstyle=\ttfamily,
	numbers=left,
	stepnumber=1,
	showstringspaces=false,
	tabsize=1,
	breaklines=true,
	breakatwhitespace=false,
	showstringspaces=false,
	breaklines=true,
	frameround=ffff,
	frame=single,
	autogobble=true
}

\title{Zadania podstawy pythona}
\author{Maciej Puchała}

\begin{document}
	
	\maketitle
	\medskip
	\begin{Large}
		\textbf{Funkcja print()}
	\end{Large}
	
	\begin{enumerate}
		
		\item
		\begin{Large}
			Zadanie 1.
		\end{Large}
		\par
		Napisz program wypisujący dowolne dwa komunikaty na ekranie.
		
		
		\item
		\begin{Large}
			Zadanie 2.
		\end{Large}
		\par
		Przypisz tekst "Hello World!" do zmiennej my\_text oraz wypisz ja na ekranie.
		\begin{lstlisting}[language=Python]		
			my_text=""
			print(my_text)
		\end{lstlisting}
		
		\par
		\item 
		\begin{Large}
			Zadanie 3.
		\end{Large}
		\par
		Funkcja print() może przyjmować wiele argumentów rożnego typu. 
		Przetestuj jej możliwość podając wiele argumentów rożnych typów np. tekst, liczby itp.
		
		\medskip
		\begin{Large}
			\textbf{Zmienne}
		\end{Large}
		\par
		\item 
		\begin{Large}
			Zadanie 4.
		\end{Large}
	\par
	Przypisz do zmiennej glass\_of\_water wartość 3.
	\begin{lstlisting}[language=Python]
		glass_of_water=
		
		print("I drank", glass_of_water, "glasses of water today.")
	\end{lstlisting}

	\par
	\item 
	\begin{Large}
		Zadanie 5.
	\end{Large}
	\par
	Zobacz co sie stanie gdy do naszej zmiennej glass\_of\_water przypiszemy nową wartość.
	\begin{lstlisting}[language=Python]
		#Fill the print function so it prints glass_of_water
		
		glass_of_water=3
		
		glass_of_water=glass_of_water + 1
		
		
		print()
	\end{lstlisting}
	\medskip
	\begin{Large}
		\textbf{Typy danych}
	\end{Large}
	\par
	
	\item 
	\begin{Large}
		Zadanie 6.
	\end{Large}
	\par
	Sprawdźmy różne typy danych dostępne w pythonie.
	\par
	\bigskip
	a) stwórz zmienną z liczbą całkowita. Następnie wypisz ja na ekranie razem z jej typem(użyj funkcji type())
	\par
	\bigskip
	b) stwórz zmienną z liczbą zmiennoprzecinkową np. zaokrągloną liczbą PI. Następnie wypisz ja na ekranie razem z jej typem(użyj funkcji type())
	\par
	\bigskip
	c) stwórz zmienną z tekstem. Następnie wypisz ja na ekranie razem z jej typem(użyj funkcji type())
	\par
	\bigskip
	d) do jednej z stworzonych zmiennych przypisz wartość innego typu i zobacz co się stanie.
	\par
	\bigskip
	e)stwórz zmienną z wartościa logiczną. Następnie wypisz ja na ekranie razem z jej typem(użyj funkcji type())
	\begin{lstlisting}[language=Python]
		#Assign True or False to the variable below then print it.
		
		staying_alive=
		print(staying_alive)
		
	\end{lstlisting}

	\medskip
	\begin{Large}
		\textbf{Konwersja typów}
	\end{Large}
	\item
	\begin{Large}
		Zadanie 7.
	\end{Large} 
	\par
	Przekonwertuj tekst(string) na liczbę całkowitą(int).
	\begin{lstlisting}[language=Python]
		#my_grade variable is a string (because it's in quotes). On line 9, convert it to an integer.
		
		my_grade="10"
		
		answer_5=
		
		
		print(answer_5)
	\end{lstlisting}

	\item
	\begin{Large}
		Zadanie 8.
	\end{Large} 
	\par
	Przekonwertuj liczbe zmiennoprzecinkową(float) na liczbę całkowitą(int).
	\begin{lstlisting}[language=Python]
	#my_temp variable is a float (because it has decimals). convert it to an integer.
	
	my_temp=97.70
	
	answer_6=
	
	
	print(answer_6)
	\end{lstlisting}

		\item
	\begin{Large}
		Zadanie 9.
	\end{Large} 
	\par
	Przekonwertuj liczbę zmiennoprzecinkową(float) na tekst(string).
	\begin{lstlisting}[language=Python]
		#my_temp variable is a float (because it has decimals). convert it to an integer.
		
		my_temp=97.70
		
		answer_6=
		
		
		print(answer_6)
	\end{lstlisting}

\medskip
\begin{Large}
	\textbf{Operatory}
\end{Large}

	\item 
	\begin{Large}
		Zadanie
	\end{Large}
	\par
	Operator przypisania. Przypisz do zmiennej lst liste kolorów ["red","green","blue"]
	\begin{lstlisting}
		
		lst=
		
		
		print(lst)
		
	\end{lstlisting}

	\item 
	\begin{Large}
		Zadanie
	\end{Large}
	\par
	Operator dzielenia. Przypisz wynik dzielnia zmiennej a przez zmieną b do zmiennej result.
	\begin{lstlisting}
		a=10
		b=3
		
		
		result=
		
		
		print(result)
		
	\end{lstlisting}

	\item 
	\begin{Large}
		Zadanie
	\end{Large}
	\par
	Operator przypisania z dodawaniem +=. Za pomocą operatora "+=" dodaj wartość 100 do zmiennej "speed".
	\begin{lstlisting}
		speed=750
			
		
		
		print(speed)
		
	\end{lstlisting}

	\item 
\begin{Large}
	Zadanie
\end{Large}
\par
Operator modulo(reszta z dzielnia) \%. Za pomocą operatora modulo przypisz reszte z dzielenia 1000/400 do zmiennej "remainder".
\begin{lstlisting}
	remainder=
	
	
	
	
	print(remainder)
	
\end{lstlisting}

	\item 
\begin{Large}
	Zadanie
\end{Large}
\par
Operator potęgowania **. Za pomocą operatora potęgowania przypisz do zmiennej p\_result kwadrat liczby 11111
\begin{lstlisting}
	p_result=
	
	
	
	
	
	print(p_result)
	
	
\end{lstlisting}

	\item 
\begin{Large}
	Zadanie
\end{Large}
\par
Operator porównania ==. Za pomocą operatora porównania sprawdź czy wartośc zmiennej "a" jest rowna zmiennej "b"
\begin{lstlisting}
	a = 450
	b = 500
	
	is_equal = 
	
	print(is_equal)
	
	
\end{lstlisting}

\medskip
\begin{Large}
	\textbf{Operacje na napisach}
\end{Large}

	\item 
\begin{Large}
	Zadanie
\end{Large}
\par
Do zmiennej napis przypisz wartość "It's always darkest before dawn."
\begin{lstlisting}
	napis=""
	
	print(napis)
	
	
\end{lstlisting}

\item 
\begin{Large}
	Zadanie
\end{Large}
\par
Poniżej mamy dwie zmienne z napisami. połącz je ze sobą w jeden napis a następnie wyświetl. zrób to na co najmniej 2 sposoby.
\begin{lstlisting}
	witaj = "Witaj "
	imie = "Jozek"
\end{lstlisting}

	\item 
\begin{Large}
	Zadanie
\end{Large}
\par
Uzywając pierwszej, drugiego i ostatniego znaku napisu ze zmiennej napis stwórz nowy napis.
\begin{lstlisting}
	napis="It's always darkest before dawn."
	
	#Type your answer here.
	
	ans_1=
	
	print(ans_1)
	
	
\end{lstlisting}

	\item 
\begin{Large}
	Zadanie
\end{Large}
\par
Zamień znak "." na znak "!"
\begin{lstlisting}
	napis="It's always darkest before dawn."
	
	#Type your code here.
	
	
	
	print(napis)
	
	
\end{lstlisting}

	\item 
\begin{Large}
	Zadanie
\end{Large}
\par
Nadpisz zmienną napis do samej siebie tak aby wszytskie znaki były małymi literami.
\begin{lstlisting}
	napis="EVERY Strike Brings Me Closer to the Next Home run."
	# Type your code here.
	
	
	
	
	print(napis)
\end{lstlisting}

	\item 
\begin{Large}
	Zadanie
\end{Large}
\par
Nadpisz zmienną napis do samej siebie tak aby wszytskie znaki były WIELKIMI literami.
\begin{lstlisting}
	napis="don't stop me now."
	# Type your code here.
	
	
	
	print(napis)
\end{lstlisting}

	\item 
\begin{Large}
	Zadanie
\end{Large}
\par
Czy napis zaczyna sie od litery "A"?\\
przypisz wartość logiczną odpowiedzi do zmiennej ans\_1.
\begin{lstlisting}
	napis="There are no traffic jams along the extra mile."
	#  Type your code here.
	
	ans_1=
	
	
	print(ans_1)
\end{lstlisting}

	\item 
\begin{Large}
	Zadanie
\end{Large}
\par
Czy napis kończy się "."?\\
przypisz wartość logiczną odpowiedzi do zmiennej ans\_1.
\begin{lstlisting}
	napis="There are no traffic jams along the extra mile."
	#  Type your code here.
	
	
	ans_1=
	
	
	print(ans_1)
\end{lstlisting}


	\item 
\begin{Large}
	Zadanie
\end{Large}
\par
a) używając metody .index(), znajdź indeks na którym znajduje się znak "v".
\begin{lstlisting}
	napis="The best revenge is massive success."
	
	#  Type your code here.
	
	ans_1=
	
	
	print(ans_1)
\end{lstlisting}
\par
b) Używając metody .find() znajdź indeks na którym znajduje się znak "m".
\begin{lstlisting}
	napis="The best revenge is massive success."
	
	#  Type your code here.
	
	ans_1=
	
	
	print(ans_1)
\end{lstlisting}
\par
c) Jaka jest różnica między tymi dwoma metodami?(spróbujcie znaleźć znak którego nie ma w napisie)

	\item 
\begin{Large}
	Zadanie
\end{Large}
\par
Który znak występuje częściej w napisie? "a" czy "o"?
\begin{lstlisting}
	napis="People often say that motivation doesn't last. Well, neither does bathing.  That's why we recommend it daily."
	#  Type your code here.
	
	ans_1=
	
	ans_2=
	
	
	print("count of a is: ", ans_1, " count of o is: ", ans_2)
	
\end{lstlisting}

	\item 
\begin{Large}
	Zadanie
\end{Large}
\par
Wypisz długość napisu ze zmiennej napis.
\begin{lstlisting}
	napis="1.975.000"
	
	#  Type your code here:
	
	ans_1=
	
	
	print(ans_1)
	
\end{lstlisting}

\medskip
\begin{Large}
	\textbf{Cięcie(Slicing)}
\end{Large}

\item 
\begin{Large}
	Zadanie
\end{Large}
\par

a) Przekrój słowo ze zmiennej wrd do pierwszego wystąpienia litery "a" (Tosc).
\par
b) Przekrój słowo wrd tak aby otrzymać "cana".
\par
c) Przekrój wrd tak aby otrzymać napis "can".
\par
d) Przekrój słowo wrd tak aby otrzymać co drugi znak.
\par
e) Przekrój słowo wrd tak aby otrzymać co drugi znak bez pierwszego i ostatniego znaku.
\par
f) Czy możesz przeciąć słowo wrd tak aby było w odwrotnej kolejności bez używania metody reverse?(anacsoT)
\begin{lstlisting}
	wrd="Toscana"
	#Type your answer here.
	
	ans_1=
	
	
	print(ans_1)
\end{lstlisting}






\medskip
\begin{Large}
	\textbf{Struktury danych}
\end{Large}

	\item
	\begin{Large}
		Zadanie 10.
	\end{Large} 
	\par
	a) Stwórz pustą listę oraz wypisz jej typ.
	\begin{lstlisting}[language=Python]
	gift_list=
	
	answer_1=
	
	
	print(answer_1)
	
	\end{lstlisting}
\par
b) Stwórz pusty słownik oraz wypisz jego typ.

	\begin{lstlisting}[language=Python]
		#Create an empty dictionary on line 3 and assign its type on line 4
		
		grocery_items=
		
		answer_2=
		
		
		print(answer_2)
	\end{lstlisting}
\par
c) Stwórz pustą krotkę oraz wypisz jej typ.

\begin{lstlisting}[language=Python]
	#Create an empty tuple on line 3 and assign its type on line 4
	
	bucket_list=
	
	answer_2=
	
	
	print(answer_2)
\end{lstlisting}

	\item 
	\begin{Large}
		Zadanie 11.
	\end{Large}
	\par
	a) Stwórz listę z wartościami oraz ja wypisz.
	\par
	b) Stwórz słownik z wartościami oraz go wypisz.
	\par
	c) Stwórz krotke z wartościami oraz ja wypisz.
	
	\medskip
	\begin{Large}
		\textbf{Listy}
	\end{Large}
	\item 
	\begin{Large}
		Zadanie 12.
	\end{Large}
	\par
	Przypisz pierwszy element z listy do zmiennej answer\_1.
	\begin{lstlisting}[language=Python]
		lst=[11, 100, 99, 1000, 999]
		answer_1=
		
		print(answer_1)
	\end{lstlisting}
	
	\item 
	\begin{Large}
		Zadanie 13.
	\end{Large}
	\par
	Wypisz drugi element listy bezpośrednio w funkcji print.
	\begin{lstlisting}
		lst=[11, 100, 101, 999, 1001]
		
		print()
		
	\end{lstlisting}
	
	\item 
	\begin{Large}
		Zadanie 14.
	\end{Large}
	\par
	Przypisz ostatni element listy do zmiennej answer\_1.
	\begin{lstlisting}
		lst=[11, 100, 101, 999, 1001]
		#Type your answer here.
		
		answer_1=
		
		print(answer_1)
	\end{lstlisting}

	\item
	\begin{Large}
		Zadanie
	\end{Large}
	\par
	Metoda .append pozwala dodawać elementy do listy.
	Dodaj do listy napis "pajamas" za pomocą metody append. 
	\begin{lstlisting}
		gift_list=['socks', '4K drone', 'wine', 'jam']
		# Type your code here.
		
		
		
		print(gift_list)
	\end{lstlisting}

	\item
	\begin{Large}
		Zadanie 15.
	\end{Large}
	\par
	W listach może znajdować się wiele różnych typów danych, możesz nawet dodać nową liste jako element list. Nazywane to jest zagnieżdżaniem danych.
	\par
	W 3 linii dodaj liste  ["socks", "tshirt", "pajamas"] na koniec listy gift\_list.
	\begin{lstlisting}
		gift_list=['socks', '4K drone', 'wine', 'jam']
		# Type your code here.
		
		
		
		print(gift_list)
	\end{lstlisting}

	\item
\begin{Large}
	Zadanie
\end{Large}
\par
Metoda .insert() pozwala wstawić element w konkretnym miejscu listy(na konkretnym indeksie).
\par
Za pomocą metody .insert() dodaj napis "slippers" na 3 miejscu listy.
\begin{lstlisting}
gift_list=['socks', '4K drone', 'wine', 'jam']
# Type your code here.



print(gift_list)
\end{lstlisting}

	\item
\begin{Large}
	Zadanie
\end{Large}
\par
Przy pomocy metody .index() można znaleźć numer indeksu na którym znajduje się dany element.
\par
Do zmiennej answer\_1 przypisz numer indeksu na ktorym znajduje się wartość 8679.
\begin{lstlisting}	
	lst=[55, 777, 54, 6, 76, 101, 1, 2, 8679, 123, 99]
	#  Type your code here.
	
	answer_1=
	
	print(answer_1)
\end{lstlisting}

	\item
\begin{Large}
	Zadanie
\end{Large}
\par
Używając metody remove usuń ostatni element z listy.
\begin{lstlisting}	
	lst=[55, 777, 54, 6, 76, 101, 1, 2, 8679, 123, 99]
	#  Type your code here.
	
	lst.remove(99)
	
	
	print(lst)
\end{lstlisting}

	\item
\begin{Large}
	Zadanie
\end{Large}
\par
Używając metody .reverse() odwróć kolejność elementów w liscie.
\begin{lstlisting}	
lst=[55, 777, 54, 6, 76, 101, 1, 2, 8679, 123, 99]
#  Type your code here.



print(lst)
\end{lstlisting}

	\item
\begin{Large}
	Zadanie
\end{Large}
\par
Używając metody .count() sprawdź ile razy 6 występuje w liście.
\begin{lstlisting}	
	lst=[55, 6, 777, 54, 6, 76, 101, 1, 6, 2, 6]
	#  Type your code inside print() function.
	
	answer_1=
	
	print(answer_1)
	
\end{lstlisting}


	\item
\begin{Large}
	Zadanie
\end{Large}
\par
Wypisz sume wszystkich elementów z listy lst.
\begin{lstlisting}	
	lst=[55, 6, 777, 54, 6, 76, 101, 1, 6, 2, 6]
	
	#  Type your code on line 4:
	answer_1=
	
	print(answer_1)
	
	
\end{lstlisting}


	\item
\begin{Large}
	Zadanie
\end{Large}
\par
a)Znajdź najmniejszy element listy.
\par
b)znajdź największy element listy
\begin{lstlisting}	
	lst=[55, 6, 777, 54, 6, 76, 101, 1, 6, 2, 6]
	
	#  Type your code on line 4:
	answer_1=
	
	print(answer_1)
	
	
\end{lstlisting}


		

\medskip
\begin{Large}
	\textbf{Krotki}
\end{Large}

\item 
\begin{Large}
	Zadanie
\end{Large}
 \par 
 Wykonaj zadania 29 - 40  zamieniając listy na krotki.
 
 \medskip
 \begin{Large}
 	\textbf{Słowniki}
 \end{Large}
 
 \item\begin{Large}
 	Zadanie
 \end{Large}
	\par
	Słowniki nie mają indeksów wiec odnoszenie sie do pierwszego lub ostatniego elementu nie jest poprawne. Zamiast tego słowniki posiadają klucze a my możemy używać kluczy do pozyskiwania z nich wartości.
	\par
	a) kiedy urodził sie Plato?
	\par
	b) Zmień date urodzenia Plato z  B.C. 427 na B.C. 428.
	
	\begin{lstlisting}
		dict={"name": "Plato", "country": "Ancient Greece", "born": -427, "teacher": "Socrates", "student": "Aristotle"}
		answer_1=
		
		print(answer_1)
	\end{lstlisting}

	\item
	\begin{Large}
		Zadanie
	\end{Large}
	\par
	Słowniki także mogą zawierać zagnieżdżone dane. Dodaj do słownika klucz "work" z poniższą listą.\\
	"work": ["Apology", "Phaedo", "Republic", "Symposium"]
	\begin{lstlisting}
		dict={"name": "Plato", "country": "Ancient Greece", "born": -427, "teacher": "Socrates", "student": "Aristotle"}
		
		#Type your answer below.
		
		
		print(dict)
	\end{lstlisting}

	\item 
	\begin{Large}
		Zadanie
	\end{Large}
	\par
	Dodaj 2 do wartości "son's height".
	\begin{lstlisting}
		dict={"son's name": "Lucas", "son's eyes": "green", "son's height": 32, "son's weight": 25}
		
		#Type your answer below.
		
		
		print(dict)
		
	\end{lstlisting}

	\item 
	\begin{Large}
		Zadanie
	\end{Large}
	\par
	Przy pomocy metody .items() wygeneruj liste krotek gdzie w kazdej krotce bedzie klucz i wartość.
	\begin{lstlisting}
		dict={"son's name": "Lucas", "son's eyes": "green", "son's height": 32, "son's weight": 25}
		
		#Type your answer below.
		ans_1= dict.items()
		
		print(ans_1)
		
	\end{lstlisting}
 
 	\item 
 	\begin{Large}
 		Zadanie
 	\end{Large}
 	\par
 	Zapoznaj sie z metodą .get() i wykonaj następujące zadania:
 	\par
 	a) wypisz wartość dla klucza "son's eyes"
 	\par
 	b) Spróbuj odczytać wartośc dla klucza  "son's age" a jeżeli klucz nie istnieje zwróć wartość "2"
 	
 	\begin{lstlisting}
 		dict = {"son's name": "Lucas", "son's eye color": "green", "son's height": 32, "son's weight": 25}
 		
 		#Type your answer inside the print.
 		ans_1=
 		
 		print (ans_1)
 	\end{lstlisting}
 
 \item 
 \begin{Large}
 	Zadanie
 \end{Large}
	\par
	Metoda .clear() służy do czyszczenia słownika. Wypróbuj ją.
	\begin{lstlisting}
		dict={"son's name": "Lucas", "son's eye color": "green", "son's height": 32, "son's weight": 25}
		
		#clear the dictionary here then print it.
		
		
		print(dict)
	\end{lstlisting}
	
	\item 
	\begin{Large}
		Zadanie
	\end{Large}
	\par
	 Za pomocą funkcji len() sprawdź ile kluczy znajduje sie w słowniku.
	 \begin{lstlisting}
	 	dict={"son's name": "Lucas", "son's eye color": "green", "son's height": 32, "son's weight": 25}
	 	
	 	#Write your answer here.
	 	ans_1=
	 	
	 	print(ans_1)
	 \end{lstlisting}
 
 \medskip
 \begin{Large}
 	\textbf{Wprowadzanie danych}
 \end{Large}
	\item 
	\begin{Large}
		Zadanie
	\end{Large}
	\par
	Używając funkcji input() poproś użytkownika o imię.
	
	\begin{lstlisting}
		#Type your answer here.
		
		ans_1=
		
		print("Hello!, " + ans_1)
	\end{lstlisting}

	\item 
	\begin{Large}
		Zadanie
	\end{Large}
	\par
	Poproś użytkownika o dane numeryczne np. wiek.
	\begin{lstlisting}
		#Type your code here.
		ans_1=
		
		
		print(type(ans_1))
	\end{lstlisting}
	
	\item 
	\begin{Large}
		Zadanie
	\end{Large}
	\par
	Poproś użytkownika o wpisanie obecnego roku i wypisz rok za 50 lat.
	
	\item 
	\begin{Large}
		Zadanie
	\end{Large}
	\par
	Stwórz konwerter który zapyta o ilość dni i przekonwertuje je na lata.
	\\ załóżmy, że rok ma zawsze 365 dni.
	\begin{lstlisting}
		# Type your answer here.
		
		message=
		
		result=
		
		
		print(result)
	\end{lstlisting}

\item 
\begin{Large}
	Zadanie
\end{Large}
\par
Stwórz konwerter który przeliczy mile na kilometry.
\\ załóżmy, że rok ma zawsze 365 dni.
\begin{lstlisting}
	# Type your answer here.
	
	message=
	
	result=
	
	
	print(result)
\end{lstlisting}

\medskip
\begin{Large}
	\textbf{Instrukcje sterujące}
\end{Large}

\item 
\begin{Large}
	Zadanie
\end{Large}
\par
Napisz program ktory przyjmie od użytkownika imię i jeżeli bedzie to "Bond" to wypisze w konsoli "Welcome on board 007."  w przeciwnym wypadku wypisze "Good morning NAME"(NAME zamień na wprowadzone imie.)

\item 
\begin{Large}
	Zadanie
\end{Large}
\par
Napisz program który sprawdzi czy wprowadzona liczba jest parzysta. i Jeżeli jest wypisze "Parzysta" w przeciwnym wypadku wypisze "nieparzysta".

\medskip
\begin{Large}
	\textbf{Pętle}
\end{Large}

\item 
\begin{Large}
	Zadanie
\end{Large}
\par
Napisz pętle while która doda do siebie wszystkie liczby od 0 do 100 włącznie.
\begin{lstlisting}
	counter=0
	total=0
	
	#Construct your while loop here.
	
	
	
	
	
	print(total)
\end{lstlisting}

\item\begin{Large}
	Zadanie
\end{Large}
\par
Używając pętli while, funkcji len(), oraz instrukcji warunkowej sprawdź czy w danej liście znajduje się wartość 100 jezeli tak to przypisz do zmiennej my\_message wiadomość na jakim indeksie sie znajduje np. "There is a 100 at index no: 5".
\begin{lstlisting}
	lst=[10, 99, 98, 85, 45, 59, 65, 66, 76, 12, 35, 13, 100, 80, 95]
	
	my_message=""
	
	#Type your code here.
	
	
	
	
	print(my_message)
\end{lstlisting}

\item 
\begin{Large}
	Zadanie
\end{Large}
\par
Używając pętli whiel i instrukcji warunkowych napisz program który stworzy nową listę napisów bez napisów pustych "".
\begin{lstlisting}
	lst1=["Joe", "Sarah", "Mike", "Jess", "", "Matt", "", "Greg"]
	
	
	#Type your code here.
	
	
	
	
	
	
	
	print(name_adder(lst1))
	
\end{lstlisting}

\item 
\begin{Large}
	Zadanie
\end{Large}
\par
używając pętli for stwórz program który wypisze każdy element z listy w osobnym wierszu.

\begin{lstlisting}
	
	lst=["koala", "cat", "fox", "panda", "chipmunk", "sloth", "penguin", "dolphin"]
	#Type your answer here.
	
\end{lstlisting}

\item 
\begin{Large}
	Zadanie
\end{Large}
\par
używając pętli for stwórz program który wypisze tekst "Hello!,"  + każde imie z listy tj. "Hello!, Sam"

\begin{lstlisting}
	
lst=["Sam", "Lisa", "Micha", "Dave", "Wyatt", "Emma", "Sage"]
#Type your code here.
	
\end{lstlisting}

\item 
\begin{Large}
	Zadanie
\end{Large}
\par
Stwórz licznik do którego w każdym obiegu pętli będziesz dodawać 1. Czy wiesz ile razy wykona się pętla?
\begin{lstlisting}
	str="Civilization"
	
	c=0
	for i in str:
	#Type your answer here.    
	
	
	print(c)
	
\end{lstlisting}

\item 
\begin{Large}
	Zadanie
\end{Large}
\par
Używając pętli for oraz metody .append() dodaj przedrostek Dr. d każdego elementu lst1.
\begin{lstlisting}
	lst1=["Phil", "Oz", "Seuss", "Dre"]
	lst2=[]
	#Type your answer here.
	
	
	
	
	print(lst2)
\end{lstlisting}

\item 
\begin{Large}
	Zadanie
\end{Large}
\par
Używając pętli for oraz metody .append() do listy lst2 dodaj kwadrat każdego elementu z lst1.
\begin{lstlisting}
	lst1=[3, 7, 6, 8, 9, 11, 15, 25]
	lst2=[]
	#Type your answer here.
	
	
	
	print(lst2)
\end{lstlisting}

\item 
\begin{Large}
	Zadanie
\end{Large}
\par
Używając pętli for napisz program który do listy lst2 doda tylko dodatnie liczby z listy lst1.
\begin{lstlisting}
	lst1=[111, 32, -9, -45, -17, 9, 85, -10]
	lst2=[]
	#Type your answer here.
	
	
	
	print(lst2)
\end{lstlisting}

\item 
\begin{Large}
	Zadanie
\end{Large}
\par
Używając pętli for napisz program który, do listy lst doda wartość ze słownika dict -1000, jeżeli wartość w słowniku jest większa od 1000, np jeżeli wartość w słowniku jest równa 1500 to do listy powinien zostać dodany element 500.
\begin{lstlisting}
	dict={"z1":900, "t1": 1100, "p1": 2300, "r1": 1050, "k1": 3200, "g1": 400}
	lst=[]
	#Type your answer here.
	
	
	
	print(lst)
\end{lstlisting}

\item 
\begin{Large}
	Zadanie
\end{Large}
\par
Napisz program który do listy lst2 dooda typ dla każdego elementu z listy lst1. np. lst1 = ["hi",1] to lst2 = [type str, type int]

\begin{lstlisting}
	lst1 = [3.14,66,"Teddy Bear",True,[],{}]
	lst2=[]
	#Type your answer here.
	
	
	
	print(lst2)
	
\end{lstlisting}
\medskip
\begin{Large}
	\textbf{List Comprehension}
\end{Large}
\item 
\begin{Large}
	Zadanie
\end{Large}
\par
Stwórz identyczna listę przy pomocy list comprehension. czyli lst1 powinna byc taka sama jak lst2
\begin{lstlisting}
	lst1=[1,2,3,4,5]
	
	#Type your answer here.
	
	lst2=
	
	print(lst2)
\end{lstlisting}

\item 
\begin{Large}
	Zadanie
\end{Large}
\par
Stwórz listę z elementów od 1200 do 2000 z krokiem 130 używając list comprehension.
\begin{lstlisting}
	#Type your answer here.
	
	rng=
	
	lst=[]
	
	print(lst)
	
\end{lstlisting}

\item 
\begin{Large}
	Zadanie
\end{Large}
\par
Użyj list comprehension aby stworzyć nową liste ale dodaj 6 do kazdego elementu listy lst1, np. lst1 = [1,2,3] lst2= [7,8,9] 
\begin{lstlisting}

#Type your answer here.

lst1=[44,54,64,74,104]

lst2=[]


print(lst2)
	
\end{lstlisting}

\item 
\begin{Large}
	Zadanie
\end{Large}
\par
Użyj list comprehension aby stworzyć nową liste z kwadratów każdego elementu listy lst1 ale tylko jeżeli jest on większy od 50.np. lst1 =[1,2,10] to lst2 = [100]
\begin{lstlisting}
	
	lst1=[2, 4, 6, 8, 10, 12, 14]
	
	#Type your answer here.
	
	lst2=[]
	
	
	print(lst2)
	
\end{lstlisting}

\item 
\begin{Large}
	Zadanie(bonus)
\end{Large}
\par
Mamy słownik w którym kluczem są typy pojazdów a wartością ich waga. Stwórz listę pojazdów których waga jest mniejsza od 5000 kg, a także ich nazwy będą w liście w wielkich liter.
\begin{lstlisting}
	dict={"Sedan": 1500, "SUV": 2000, "Pickup": 2500, "Minivan": 1600, "Van": 2400, "Semi": 13600, "Bicycle": 7, "Motorcycle": 110}
	
	#Type your answer here.
	
	lst=[]
	
	print(lst)
\end{lstlisting}

\medskip
\begin{Large}
	Mini projekty
\end{Large}

\item 
\begin{Large}
	Wisielec
\end{Large}
\par
a) stwórz zmienną na słowa z których program będzie mógł wybrać jedno (na początku możne to być stałe jedno słowo)
\par
b) stwórz zmienną która będzie przechowywała liczbę szans np. 5 po 5 niepoprawnych zgadnięciach litery gra się kończy.
\par
c) stwórz główną pętle gry która będzie działała tak długo az liczba szans się nie wyczerpie lub słowo nie zostanie w pełni odkryte.
\par
d) wewnątrz pętli poproś o podanie litery, sprawdź czy litera znajduje sie w słowie i wypisz miejsca w których sie znajduje np. "\_\_\_a\_\_\_a\_\_\_" jeżeli sie nie znajduje zmniejsz liczbę szans o 1, na końcu sprawdź czy słowo nie jest kompletne i jeżeli jest to zakończ grę.

\item 
\begin{Large}
	Kółko i krzyżyk
\end{Large}
\par
a) Stwórz zmienna(liste) która będzie przechowywac dane dla każdej komórki planszy czyli o,x lub pusty napis. A także stwórz zmienną która będzie przechowywać symbol obecnego gracza o lub x
\par
b) narysuj pustą plansze do gry w konsoli za pomocą funkcji print.
\par
c) Stwórz główną pętle gry, która będzie działała az jeden z graczy nie wygra lub nie skończy się miejsce na planszy(remis).
\par
d) w pętli poproś gracza o wpisanie pozycji w której chce wstawic swój symbol. wstaw symbol w odpowiednie miejsce listy. następnie wyrysuj plansze z naniesionymi znakami. sprawdź czy gracz nie wygrał lub nie skonczyło sie miejsce na planszy. i zmień gracza.
\par
e)poinformuj o zwycięstwie lub remisie i zakończ pętle

\item 
\begin{Large}
	Konsolowa symulacja bazy danych (np. baza wydatków, baza studentów z ocenami itp.)
\end{Large}
\par
a) stwórz kolekcie która będzie imitować bazę danych wybierz która będzie ci najbardziej pasować.(lista, słownik, krotka?, zbiór?)
\par
b) Stwórz petle w której zapytasz użytkownika co chce zrobic.np, dodac nowy wpis, usunąc wpis, edytowac wpis itp.
\par
c) następnie w zależność od wcześniejszego wyboru zapytasz o dane do wpisania lub o wiersz do usunięcia lub zmiany.(jak będzie to wyglądać zależy jakiego typu kolekcje wybierzecie.)
\par
d) po wykonaniu operacji powinniśmy wrócić do 1 menu.
\par
e) umożliwcie także wyjście z wszego programu przez użytkownika np poprzez wpisanie exit lub odpowiedniego wyboru z opcji w 1 menu.
\par
Bonus można przed samym zamknięciem zapisać baze do pliku. (tego jeszcze nie było na zajęciach) jak ktos by bardzo chcial to zrobic to tutaj jest link jak otwierać pliki i do nich pisać w pythonie: \href{http://davinci.mimuw.edu.pl/samouczek-python-python/operacje-na-plikach}{Operacje na plikach}

\medskip
\begin{Large}
	Funkcje
\end{Large}
\par
\item 
\begin{Large}
	Zadanie
\end{Large}
\par
Zdefiniuj funkcje f\_1 która wypisze tekst "Hello world!".
\begin{lstlisting}
	def f_1():
	
		
	f_1()
\end{lstlisting}

\par
\item 
\begin{Large}
	Zadanie
\end{Large}
\par
Zdefiniuj funkcje f\_1 która wypisze tekst "Hello world!". a następnie przypisz jej wynik do zmiennej ans\_1.
\begin{lstlisting}
	def f_1():
	print("Hello World!")
	
		
	ans_1=
	print(ans_1)
\end{lstlisting}

\par
\item 
\begin{Large}
	Zadanie
\end{Large}
\par
Zdefiniuj funkcje f\_1 która zwróci tekst "Hello world!". a następnie przypisz jej wynik do zmiennej ans\_1.
\begin{lstlisting}
	def f_1():
	
	
	ans_1=f_1()
	print(ans_1)
\end{lstlisting}

\par
\item 
\begin{Large}
	Zadanie
\end{Large}
\par
Zdefiniuj funkcje f\_1 która wypisze i zwróci tekst "Hello world!". a następnie przypisz jej wynik do zmiennej ans\_1.
\begin{lstlisting}
	def f_1():
	
	
	ans_1=f_1()
	print(ans_1)
\end{lstlisting}

\par
\item 
\begin{Large}
	Zadanie
\end{Large}
\par
Zdefiniuj funkcje f\_1 która zawsze zwroci wartość 100.
\begin{lstlisting}
	# Type your answer here.
	
	
	
	
	
	print(f_1(999))
\end{lstlisting}


\par
\item 
\begin{Large}
	Zadanie
\end{Large}
\par
Zdefiniuj funkcje f\_1 która pryzmie wartość liczbowa jako parametr a następnie ja zwróci.
\begin{lstlisting}
# Type your answer here.





print(f_1(44))
\end{lstlisting}

\par
\item 
\begin{Large}
	Zadanie
\end{Large}
\par
Zdefiniuj funkcje f\_1 która przyjmie listę jako parametr oraz zwróci tą listę w odwrotnej kolejności.
\begin{lstlisting}
	lst=[100,200,300,400,500]
	
	
	
	
	
	
	f_1(lst)
	print(lst)
\end{lstlisting}

\par
\item 
\begin{Large}
	Zadanie
\end{Large}
\par
Zdefiniuj funkcje f\_1 która zapyta użytkownika o imie i wypisze tekst "Hello {name}".
\begin{lstlisting}
	# Type your answer here.
	
	
	
	f_1()
	
\end{lstlisting}


\par
\item 
\begin{Large}
	Zadanie
\end{Large}
\par
Zdefiniuj 2 funkcje f\_1 oraz f\_2. pierwsza funkcja przyjmuje liczbę jako parametr i zwraca ta liczbę+5. Druga funkcja przyjmie liczbę jako parametr i zwróci wynik f\_1 pomnożony przez 2.
\begin{lstlisting}
	# Type your answer here.
	
	
	
	print(f_2(4))
	
\end{lstlisting}

\medskip
\begin{Large}
	Moduły
\end{Large}
\par

\item 
\begin{Large}
	Zadanie
\end{Large}
\par
Napisz prosty program ktory wygeneruje:
\par
a) losoowy ciag alfabetyczny znaków (zobacz tablice ascii),
\par
b) losowa liczbe z danego przedziału
\par
c)losową wielokrotność liczby 7 z zakresu 0-70.
\par
uzyj random.randint(), random.choice(),
do losowej litery można tez użyć modułu string

\item 
\begin{Large}
	Zadannie
\end{Large}
\par
Napisz program który przetasuje elementy z danej listy.
\begin{lstlisting}
	import random
	
	numbers = [1, 2, 3, 4, 5]
	words = ['red', 'black', 'green', 'blue']
	
	print(numbers)
	print(words)
	
	print("Shuffle")
	#Your code here
	
	print(numbers)
	print(words)
\end{lstlisting}

\item 
\begin{Large}
	Zadanie
\end{Large}
\par
Napisz program w języku Python, który utworzy listę 100 losowych liczb całkowitych i losowo wybierze 10 elementów z tej listy.
uzyj funkcji random.sample()
\par
\href{https://docs.python.org/3/library/random.html?highlight=random#module-random}{https://docs.python.org/3/library/random.html?highlight=random#module-random}

\item 
\begin{Large}
	Zadanie
\end{Large}
\par
Zamien podany string na obiekt typu datetime np.\\
wejscie:\\
date\_string = "Feb 25 2020 4:20PM"\\
wyjscie:\\
2020-02-25 16:20:00\\

\href{https://docs.python.org/3/library/datetime.html?highlight=strptime#datetime.datetime.strptime}{dokumentacja}

\begin{lstlisting}
	from datetime import datetime
	
	date_string = "Feb 25 2020 4:20PM"
	
	datetime_object = 
	
	print(type(datetime_object))
	print(datetime_object)
\end{lstlisting}

\item 
\begin{Large}
	Zadanie
\end{Large}
\par
Napisz program który odejmie tydzien od danej daty.\\
Wejscie:\\
given\_date = datetime(2020, 2, 25)\\
Wyjscie:\\
2020-02-18\\
przydatny moze okazac sie obiekt timedelta z modulu datetime.

\begin{lstlisting}
	from datetime import datetime, timedelta
	
	given_date = datetime(2020, 2, 25)
	print("Given date")
	print(given_date)
	
	days_to_subtract = 7
	res_date = 
	print("New Date")
	print(res_date)
\end{lstlisting}


\medskip
\begin{Large}
	Programowanie obiektowe
\end{Large}

\item 
\begin{Large}
	Zadanie
\end{Large}
\par
Stworz pusta klase Vehicle

\item 
\begin{Large}
	Zadanie
\end{Large}
\par
Do klasy vehicle dodaj atrybuty max\_speed oraz mileage(przebieg)

\item 
\begin{Large}
	Zadanie
\end{Large}
\par
Stworz klase pochodna od Vehicle ktora bedzie nazywac sie Bus i bedzie posiadać wszystkie pola z Vehicle.

\item 
\begin{Large}
	Zadanie
\end{Large}
\par
Dodaj do klasy Vehicle metode seating\_capacity a następnie w klasie Bus spraw aby metoda seating\_capacity miala domyślna wartość 50


\item 
\begin{Large}
	Zadanie
\end{Large}
\par
Zdefiniuj atrybut klasowy „color” z domyślną wartością White. Oznacza to, że każdy Vehicle powinien być biały.

\item 
\begin{Large}
	Zadanie
\end{Large}
\par
Utworz Klase pochodna Bus dziedzicząca po klasie Vehicle. z metoda wyliczajacą opłate(fare) dla klasy Bus powinno to byc
\[final amount = total fare + 10\%\] of the total fare. uzyj ponizszej klasy Vehicle:
\begin{lstlisting}
class Vehicle:
	def __init__(self, name, mileage, capacity):
		self.name = name
		self.mileage = mileage
		self.capacity = capacity

	def fare(self):
		return self.capacity * 100

class Bus(Vehicle):
	pass

School_bus = Bus("School Volvo", 12, 50)
print("Total Bus fare is:", School_bus.fare())
\end{lstlisting}

\item 
\begin{Large}
	Zadanie
\end{Large}
\par
Sprawdz jakiego typu jest zmienna School\_bus:
\begin{lstlisting}
class Vehicle:
	def __init__(self, name, mileage, capacity):
		self.name = name
		self.mileage = mileage
		self.capacity = capacity
	
class Bus(Vehicle):
	pass
	
School_bus = Bus("School Volvo", 12, 50)
\end{lstlisting}

\item 
\begin{Large}
	Zadanie
\end{Large}
\par
Sprawdź czy School\_bus jest także instancją klasy Vehicle.

\item 
\begin{Large}
	Zadanie
\end{Large}
\par
Mini projekt:\\
-Stwórz klase dla kart do gry.\\
-Klasa karty powinna mieć kolor (kiery, karo, trefl, pik) i wartość (A,2,3,4,5,6,7,8,9,10,J,Q,K)\\
-Stwórz klase Talii ktora zawiera 52 karty\\
-posiada metode deal ktora wyciaga karte z talii\\
-po wyjeciu karty z talii karta powinna zostac z niej usunięta\\
-posiada metode shuffle która upewnia się, że talia kart zawiera wszystkie 52 karty, a następnie losowo je przestawia.\\
	
\end{enumerate}
\end{document}